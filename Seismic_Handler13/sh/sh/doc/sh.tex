\documentstyle[a4]{article}

% file deflist.tex
%      ===========
%
% another list type for latex
%
\newcommand{\deflabel}[1]{\bf #1\hfill}%    1
\newenvironment{deflist}[1]%                2
{\begin{list}{}%                            3
{\settowidth{\labelwidth}{\bf #1}%          4
\setlength{\leftmargin}{\labelwidth}%       5
\addtolength{\leftmargin}{\labelsep}%       6
\renewcommand{\makelabel}{\deflabel}}}%     7
{\end{list}}%                               8

\newcommand{\exm}[1]{{\tt #1}}    % example
\newcommand{\cmd}[1]{{\tt #1}}    % command verb
\newcommand{\shi}[1]{{\sc #1}}    % SH identifier
\newcommand{\sht}[1]{{\sc #1}}    % SH type

\begin{document}


\title{SeismicHandler}
\author{K.\ Stammler}
\date{27 April 1992}

\maketitle

\tableofcontents


\section{General Information}

SeismicHandler (SH) is a tool for analysing digital seismograms.
It can be used for the analysis of earthquake records as well as for
examining seismogram sections in refraction seismology.  The program
was developed during the work at my PhD at the SZGRF in Erlangen.
Excluding graphic interfaces it now consists of roughly 40,000 lines
of source code, written in ANSI C.  The current version was developed
on ATARI ST/TT and has been exported to a MicroVAX and to Sun
computers, so the portability should be guaranteed.

The program uses dynamic storage allocation.  The length of the traces
SH can hold in memory is limited only by the RAM size of the computer.
The maximum number of traces in memory is currently limited to 300.
Trace editing functions such as spike removal, baseline correction
or polynomial interpolation are available.

Although SH will run on a Tektronix terminal, it can more conveniently
be used on a window system.  Currently implemented window graphics
interfaces are X--Window (tested on VAX and SUN), VWS (VAX/VMS window
system) and GEM (ATARI).  On a window system you have at least two
different windows, one dialog window for entering commands and a
second for graphic output.  SH is able to handle up to seven windows,
but you will rarely use more than three.  It is, for example,
convenient to open a separate window for particle motion diagrams.

The user interface of SH is a command language designed specifically
for processing seismic data.  Command lines are typed in interactively
or may be read from a file (command procedure).  Dialog boxes,
pop--up or pull--down menues have not been implemented due to the
incompatibility of different graphics interfaces.

There is a help text available for each SH command.  It can be
requested interactively by the \cmd{help}\ command.  If you create
new commands you can update the help library just by adding a help
text file to the help directory.



\section{The Command Line}

The command interpreter parses each command line by processing
two steps:
\begin{enumerate}
\item split up the command line in words using blanks, semicolons
      and slashes
      as terminators.  Words are separated either by one or
      more blanks or by a single semicolon or by a single
      semicolon and additional blanks or by a slash which may be
      preceded by blanks.  Two consecutive semicolons
      or two semicolons with only blanks in between denote an
      empty word.
\item translation of each word.  A description of the translation
      process is given in section \ref{sec:Translation}.
\end{enumerate}
All words which were separated by a slash ("/"), are identified
as qualifiers (see below).
Qualifiers are treated separately and do not count as parameters.
Besides these qualifiers the first word is regarded as the command
verb, the others as parameters numbered from 1 to N.  N ranges
from zero to fifteen.  Empty words (see above) result in
an empty parameter.  Note that the first parameter can
be separated from the verb by a semicolon as well.  That means
if you want to pass an empty first parameter and a second
parameter \exm{par}\ to a command verb \exm{verb}, you have to
type \exm{verb;;par} and {\em not} \exm{verb;par}, where \exm{par}\
is regarded as the first parameter.

Qualifiers are most often used as switches, using their presence
as one state of the switch, their absence as the other.  Such simple
qualifiers \exm{/SimpleQual}\ consist of the slash ("/") and
the name of the qualifier.  However, on some commands you may
pass valued qualifiers \exm{/ValuedQual=value}.  The value
\exm{value}\ is appended to the valued qualifier with a "="-character
in between.  There is a maximum number of 5 qualifiers per command
permitted.  A complicated example of a command line is

\exm{verb par1;par2 ;; par4/SimpleQual par5  par6\ \ /ValuedQual=val}.

\noindent
The exclamation character "!" is regarded as the end of the
command line.  Any characters following it are ignored and may be
used as a comment on the command line.  A command line beginning
with the exclamation character is equivalent to an empty line.



\section{Translation of Expressions}
\label{sec:Translation}

As it was mentioned in the previous section, the command parser
translates each word of the given command line if possible.
The translation process is applied if the word starts with one
of the following non--alphanumeric characters:
\verb+" # $ % _ ^ |+

The "\verb+|+"--characters are used for concatenation of
subexpressions, which may be translated itself before. The
construction \exm{|<sub1>|<sub2>|<sub3>|}, for example, results in a
concatenation of the three subexpressions \exm{<sub1>},
\exm{<sub2>} and \exm{<sub3>}.  There is a maximum of 10
subexpressions which can be concatenated within one command
line.

All others of the previously listed characters require either
an indexed or a non--indexed name to follow.  A non--indexed
name is just a string of alphanumeric characters \exm{<name>},
an indexed name consists besides the name an index string
in paranthesis \exm{<name>(<index>)}.  Now follows a list of
valid translatable expressions:

\begin{description}
\item[\exm{"<name>}]
   Expression is replaced by the current value
   of the local or global variable \exm{<name>}.  See
   section \ref{sec:Variables}\ for detailed information
   about variables in SH.  \exm{<name>}\ must be a defined
   variable name.  See command \cmd{sdef}.
\item[\exm{\#<name>}]
   If \exm{<name>}\ is a number between 1
   and 15 the expression is replaced by the value of the
   \exm{<name>}--th parameter passed to the current command
   procedure (see section \ref{sec:CmdProc}).  If \exm{<name>}\
   equals the string \exm{params}\ it is replaced by the total
   number of parameters passed to the command procedure.  Any
   other expression accesses qualifiers of the command line which
   called the current procedure.  For example, the word
   \exm{\#example}\ is replaced by
   \begin{itemize}
   \item  the string \verb+_EXISTSNOT_+ if the qualifier
      \exm{/example}\ was not specified.
   \item  the string \verb+_NOVALUE_+ if the qualifier
      \exm{/example}\ was specified without value.
   \item  the value \exm{qualvalue}\ if a valued qualifier
      \exm{/example=qualvalue}\ was specified.
   \end{itemize}
\item[\exm{\$<name>}]
   This evaluates internal variables.  An example
   is \exm{\$dsptrcs}\ which returns the current number of traces
   on the display.  A complete list of all internal variables is
   given in section \ref{sec:InternVar}.
\item[\exm{\%<file>(<line>)}]
   Returns the \exm{<line>}--th line of the text file
   \exm{<file>}.  If the parantheses and \exm{<line>}\ are omitted
   the first line of \exm{<file>}\ is returned.  A value of 0 for
   \exm{<line>}\ returns the total number of lines in \exm{<file>}.
   If the file extension is
   not specified it is assumed to be \exm{.stx} which is the
   default extension for SH text output files (\cmd{echo} command).
   Please make sure that the text file doesn't contain lines
   longer than 132 characters.  Otherwise the line counting won't
   be correct.
\item[\exm{$^\wedge$<info>(<trace>)}]
   Returns the value of the info entry \exm{<info>}\ of the
   \exm{<trace>}--th trace on display (counting from bottum up).
   If the trace number and the parantheses are omitted the first
   trace is accessed.  As an example, the expression
   \verb+^delta(3)+ is replaced by the sample distance
   of the third trace on display.  For detailed information about
   info entries see section \ref{sec:InfoEntries}.
\item[\exm{\_<info>(<start>:<end>)}]
   Such an expression is translated only if it is found in place
   of a trace list parameter.  It selects all traces which have
   an \exm{<info>}--value between \exm{<start>}\ and \exm{<end>}.
   A more detailed explanation is given in section \ref{sec:TraceAdr}.
\end{description}

In an indexed name both subexpressions (\exm{<name>}\ and
\exm{<index>}) itself can be of the above (non--indexed) type.  As
an example, the expression \exm{\%\#1("cnt)}\ is replaced by the
line number \exm{cnt}\ of a text file, whose name is passed as the
first parameter to the current command procedure.



\section{Information Entries}
\label{sec:InfoEntries}

Usually there is a lot of information about a seismogram besides
the sample data itself.  Some examples are the {\em length} of
the trace, the {\em sample distance}, the recording {\em station}
and {\em component} name.  All these information values and many more
can be stored in the q--files, the preferred data format of SH.
The two most convenient issues of SH concerning the q--file format
are (i)~the accessibility for reading and writing of each of these
information values through SH by descriptive names and (ii)~the
possibility of defining your own information entries for the
q--files (including their names).

To use the information entries you don't have to know much about
the q--file format, but you should know a little about how SH
accesses them.  The most important thing about information
entries in q--files is, that they have a type (called \shi{q--type})
and a number (called \shi{q--number}).  For your convenience
there is a
name assigned to each information entry (called \shi{infoname}), which
SH can translate into \shi{q--type}\ and \shi{q--number}\ to
identify the
information uniquely.  But there is a problem concerning the
execution speed.  If SH would read the Q-file header on each
read--access to any of the information entries, this would slow
down the program drastically.  A write--access is even worse, because
SH would have to rewrite the whole Q--header file.  For this reason
SH stores frequently used information entries for each trace in
memory.  That means that every \shi{infoname}\ does not only point to
a \shi{q--type}\ and a \shi{q--number}, but as well to a type
(called \shi{sh--type})
and an index (called \shi{sh--index}) in memory.  The
\shi{sh--index}\
is different from q--number, because the \shi{sh--index}\
controls whether
or not the information entry is stored in memory (see below).
So you can change the set of information entries hold in
memory by changing
their \shi{sh-index}\ numbers without changing the \shi{q--number}s.
It is necessary to distinct between \shi{q--type}\ and \shi{sh--type},
because there exist more \shi{sh--type}s than \shi{q--type}s.
That means, some \shi{sh--type}s do not exist on q--file and are
converted to an existing \shi{q--type}\ to store it on q--file.  For
example the \shi{sh--type}\ \sht{time}\ is stored as a \sht{string}\
\shi{q--type}.

On a read--access to an information value SH translates the
given \shi{infoname}\ to \shi{sh--type}\ and \shi{sh--index}.
By the value of \shi{sh--index}\ SH can tell whether the information
is hold in memory.  If this is the case, SH takes the information
found in memory, it doesn't touch the information in the
q--file.  If the information is not found in memory, SH determines
\shi{q--type}\ and \shi{q--number} and reads it from q--file.

Any write--access (command \cmd{set}) to information values changes
only values in memory by default.  If the information is not found
in memory, SH doesn't change anything.  Only on explicit request
(command \cmd{set/file}) SH changes information entries on file
(and in memory if it is there as well).

Some information entries do not exist on q--file but are hold
in memory.
These entries are created when the trace is read or generated
in any other way.  This is useful, for example, for entries like
minimum and maximum amplitude of a trace or any information concerning
the display, like time origin, vertical position, display attributes,
normalization factors and so on.

As already mentioned, the \shi{sh--index}\ controls the storage type of
the information entry.  There were mentioned three storage types of
entries.  These are the frequently used information
entries stored in q--file {\em and} hold in memory (called 
\shi{auto--load}\
entries, because they are loaded automatically into memory when
a trace is read in), the rather slow accessible entries stored
in q--files only (called \shi{file--only}\ entries) and the temporary
information entries, which are hold in memory only (called
\shi{mem--only}\ entries).  For each \shi{sh--type}\ there is
a different but
fixed number of entries (called \shi{maxmem}) which can be hold
in memory.
This number includes the \shi{mem--only}\ entries which are predefined
and cannot (or at least should not) be changed.  Therefore remains
a smaller number (called \shi{maxauto}) of \shi{auto--load}\
entries which
are definable by the user.  In detail the index structure is
like this:  all entries with an \shi{sh--index}\ ranging from 0 to
\shi{maxauto}-1 are \shi{auto--load}\ entries, all entries from
\shi{maxauto}\ to
\shi{maxmem}-1 are \shi{mem--only}\ entries, all entries greater or
equal
to \shi{maxmem}\ are \shi{file--only}\ entries.  Note that SH
supports only
\shi{sh--index}\ numbers smaller than a number \shi{maxfile}.
The values of the numbers \shi{maxauto}, \shi{maxmem}\ and
\shi{maxfile}\
for each entry type and all currently defined information
entries including their \shi{sh--index}\ and \shi{q--number}\
can be listed
by the SH command \exm{entry list outfile.txt}.  Then a file
\exm{outfile.txt}\ is created and typed on the current output
window of SH.

Remember that the \shi{sh--index}\
is an SH--internal number and does not appear in q--files in
any way.  For the identification of an entry, only the
\shi{q--number}\ is decisive.  This allows you to change the set of
entries which are used as \shi{auto--load}\ from one SH session to
another.

Now follows a complete list of all available entry types
(all are valid as \shi{sh--type}s and most of them are valid
\shi{q--type}s)
\begin{deflist}{\sht{integer}}
\item[\sht{long}]  \shi{sh--type}\ and \shi{q--type}.  32--bit signed 
             integer value.
\item[\sht{integer}]  \shi{sh--type}\ and \shi{q--type}.
             16--bit signed integer value.
\item[\sht{byte}]  \shi{sh--type}\ only, its converted to
             \sht{integer}\ type on q--files.  8--bit signed integer
             value.
\item[\sht{real}]  \shi{sh--type}\ and \shi{q--type}.  Real number
             in floating point or exponential format.
\item[\sht{string}]  \shi{sh--type}\ and \shi{q--type}.  Character
             string containing any printable character except
             tilde ("\verb+~+").
\item[\sht{char}]  \shi{sh--type}\ and \shi{q--type}.  Single
             printable character (no tilde "\verb+~+" permitted).
\item[\sht{time}]  \shi{sh--type}\ only, its converted to
             \sht{string}\ type on q--files.
             Absolute time specification, containing date and time.\\
             Example: 23--JUN--1989\_23:30:00.000
\item[\sht{flag}]  \shi{sh--type}\ only, its converted to \sht{char}\
             type on q--files.  Two--valued entry, possible values
             are \exm{yes}\ and \exm{no}.
\end{deflist}

Since you can create your own information entries, there are
predefined only the entries which are internally used by SH.
You should not change the \shi{sh--index}\ or the \shi{q--number}\
of any
of these predefined entries, even if you can.  If you do so,
you will slow down the program (in the best case) or crash
it (the worst case).
This is a complete list of the predefined entries:
\begin{deflist}{\shi{reduction}}
   \item[\shi{length}]  \shi{auto--load}, \shi{sh--type}\ and \shi{q--type}\
      \sht{long},
      \shi{q--number} 1.  Length of trace in
      number of samples.
   \item[\shi{alloc}]  \shi{mem--only}, \shi{sh--type}\ \sht{long}.
      Size of allocated memory for the trace in units of samples.
      Usually this is the same as LENGTH.
   \item[\shi{dspfst}]  \shi{mem--only}, \shi{sh--type}\ \sht{long}.
      Index number of first sample inside the current display
      window.  Controlled by commands \cmd{STW}\ and \cmd{DTW}.
   \item[\shi{dspcnt}]  \shi{mem--only}, \shi{sh--type}\ \sht{long}.
      Number of samples inside the current display window.
      Controlled by commands \cmd{STW}\ and \cmd{DTW}.
   \item[\shi{recno}]  \shi{mem--only}, \shi{sh--type}\ \sht{integer}.
      If the trace is read from a q--file, this entry contains
      the position number of the trace inside the file, otherwise
      the entry value is zero.
   \item[\shi{attrib}]
      \shi{mem--only}, \shi{sh--type}\ \sht{integer}.
      Number of the display attribute block for the trace.
      A display attribute block controls the output attributes
      of traces like colour, line width and line style {\em and}
      the output attributes of text like colour, size, font and
      text effects.  For details about attribute blocks see
      section \ref{sec:AttribBlocks}. The number of available
      attribute blocks depends on the implemented graphics package.
      The default attribute block is 0.
   \item[\shi{reduction}]
      \shi{mem--only}, \shi{sh--type}\ \sht{integer}.
      Reduction factor for trace plotting.  Reduces number of
      samples on display and increases output speed.  If, for
      example, the reduction factor is 3, every third point of
      the trace is plotted.  The default value of the reduction
      is 1, that means every point is plotted.
   \item[\shi{delta}]
      \shi{auto--load}, \shi{sh--type}\ and \shi{q--type}\ \sht{real},
      \shi{q--number}\ 0.  Usually this is the
      sample distance in seconds.
   \item[\shi{maxval}]
      \shi{mem--only}, \shi{sh--type}\ \sht{real}.
      Value of the maximum sample of the whole trace.  Determined
      automatically when the trace is read and after every trace
      manipulation.
   \item[\shi{minval}]
      \shi{mem--only}, \shi{sh--type}\ \sht{real}.
      Value of the minimum sample of the whole trace.  Determined
      automatically when the trace is read and after every trace
      manipulation.
   \item[\shi{norm}]
      \shi{mem--only},  \shi{sh--type}\ \sht{real}.
      Normalization factor.  Determined automatically before
      each redraw, depending on the selected normalisation mode
      (see command \cmd{norm}).
   \item[\shi{zoom}]
      \shi{mem--only}, \shi{sh--type}\ \sht{real}.
      Zoom factor entered by user via command \cmd{zoom}.  Default value
      is 1.  The total amplification factor of the trace is
      the product of \cmd{norm}\ and \cmd{zoom}.
   \item[\shi{t-origin}]
      \shi{mem--only}, \shi{sh--type}\ \sht{real}.
      Horizontal position of trace.  Given in the same units
      as \exm{delta}.  Controlled by commands \cmd{shift} and
      \cmd{al}, \cmd{beam}\ and others.
   \item[\shi{s-origin}]
      \shi{mem--only}, \shi{sh--type}\ \sht{real}.
      Vertical position of trace.  Determined automatically before
      each redraw.  Computation algorithm can be modifed with
      \cmd{yinfo}\ command.
   \item[\shi{weight}]
      \shi{mem--only}, \shi{sh--type}\ \sht{real}.
      Weight of trace when used in a \cmd{sum}\ command.  Default
      value is 1.
   \item[\shi{comment}]
      \shi{auto--load}, \shi{sh--type}\ and \shi{q--type}\ \sht{string},
      \shi{q--number}\ 0.  Comment line on trace.
   \item[\shi{station}]
      \shi{auto--load}, \shi{sh--type}\ and \shi{q--type}\ \sht{string},
      \shi{q--number}\ 1.  Station code of
      recording station.
   \item[\shi{file}]
      \shi{mem--only}, \shi{sh--type}\ \sht{string}.
      Name of input file of trace.  If the trace is generated
      in SH the string is copied from a parent trace if possible,
      otherwise it remains empty.
   \item[\shi{comp}]
      \shi{auto--load}, \shi{sh--type}\ and \shi{q--type}\ \sht{char},
      \shi{q--number}\ 0.  Component of
      recording station.
   \item[\shi{start}]
      \shi{auto--load}, \shi{sh--type}\ \sht{time}, \shi{q--type}\ string,
      \shi{q--number}\ 21.  Start date and time
      of record.
   \item[\shi{modif}]
      \shi{mem--only}, \shi{sh--type}\ \sht{flag}.
      If the trace is modified since read in from q--file, this
      flag is set to \exm{yes}, otherwise its \exm{no}.
   \item[\shi{fromq}]
      \shi{mem--only}, \shi{sh--type}\ \sht{flag}.
      If the trace is read from a q--file this flag is set to
      \exm{yes}, otherwise its \exm{no}.
\end{deflist}

When you define your own entries, you should make sure that
\begin{itemize}
\item  the \shi{q--number}\ is not used by another already defined
       entry (this is not checked by SH !)
\item  the \shi{q--number}\ is not greater or equal to \shi{maxfile}
\item  the \shi{sh--index}\ is not used by another already defined
       entry (this is checked)
\item  the \shi{sh--index}\ is not that of a \shi{mem--only}\ entry,
       that means it should not be inside the range from
       \shi{maxauto}\ to \shi{maxmem}-1
\item  the \shi{sh--index} is not greater or equal to \shi{maxfile}
\end{itemize}
Usually additional entries are defined within the startup command
file of SH \cmd{shstrtup}.  To get a complete list of the currently
defined entries, their \shi{sh--index}\ numbers and their
\shi{q--number}s, enter the command \exm{entry list outfile.txt},
which creates a file \exm{outfile.txt}\ and types it on the current
output window.  To define entries use the command \cmd{entry define}.
See \cmd{help entry}\ for detailed information about the \cmd{entry}\
command.



\section{Trace Addressing}
\label{sec:TraceAdr}

Many commands require a list of traces as input parameter.  Then
you will usually specify a number of traces of the current display.
The traces are addressed by their position number inside the
display window.  The positions are counted from the bottom up to
the top, starting at 1.  By default, the position numbers are
displayed on the left side of each trace, where the name of the
recording station and the component are given as well.  But this
trace labelling can be changed (command \cmd{trctxt}) and so the
trace numbering is not necessarily visible.  Let's use the
command \cmd{del}\ as an example.  You may use the following
list expressions in place of any parameter of type \cmd{trace list}\
(the parameter type is always given in the help text to each
command).  The delete command \cmd{del}\ takes only one parameter
which is the list of traces to be deleted.  Suppose you want to
delete the bottom trace only.  This is done by \cmd{del 1}.  After
execution the display is redrawn automatically (if the display
redraw is enabled) without the deleted first trace.  The bottom
trace which had position number 2 before the delete command, now
got the position number 1.  It is possible to specify a set of
traces by a list of position numbers, separated by commas.  For
example, to delete the first, third and fifth trace, the command
is \cmd{del 1,3,5}.  A block of consecutive numbers can be specified
by the first and last number of the block, separated by a hyphen.
So the command \cmd{del 1-6}\ deletes the first 6 traces on the
display.  Blocks and lists may be combined which means \cmd{del
1-3,7-9,14-16}\ is a valid command and deletes the traces with
the position numbers 1, 2, 3, 7, 8, 9, 14, 15, 16.  Please keep
in mind, that any such expression must not contain blanks, because
otherwise the command parser would regard the expression as more
than one parameter and the parameter passing won't be correct.
To select all traces of the display window, use the expression
\cmd{all}, which means \cmd{del all}\ deletes all displayed traces.

It is possible to change the order of the traces on display by
the \cmd{display}\ command.  For example, \cmd{display 3 1}\ puts
trace number 3 to the bottom by changing it's position to 1.
Also valid is \cmd{display 7-9 3}, which takes the traces at the
positions 7 to 9 and inserts them at positions 3 to 5.

Sometimes it is useful to remove traces from the display and
keep them in memory.  Then the traces can be redisplayed later
without reading them from file again (which takes time and you need
to remember the filename) or without computing them again, if the
traces are not read from file directly.  The command is
\cmd{hide <list>}.  \cmd{<list>}\ may be any trace list parameter
as described above.  Then the traces specified in \cmd{<list>}\
are hidden, they are not displayed any more.  They are redisplayed
by the \cmd{display}\ command.  But since the traces are not
visible, they don't have a position number which is usually needed
for addressing.  Therefore a special expression \cmd{h:all}\ is
defined which includes all hidden traces.  The command \cmd{display
h:all 3}\ displays all hidden traces starting at position number 3.
The second parameter may be omitted.  It is then assumed to be 1.
With this command all of the hidden traces are redisplayed.  To
redisplay only a subset, you need an expression which is explained
in the following.

There exists a possibility to select a subset of all traces in
memory which match a condition given for a specified info entry.
A range for this info entry is selected and all traces which have
an info entry within this range are put on the list.  The general
syntax of such an expression is \cmd{\_<info>(<start>:<end>)}.
\cmd{<info>}\ is the name of the info entry.  \cmd{<start>}\ and
\cmd{<end>}\ specify the limits of the value range.  As an example,
let's assume that you want to display only traces which have an
epicentral distance between 30$^\circ$ and 50$^\circ$.  First
all traces must be hidden, using \cmd{hide all}.  Then redisplay
all traces matching the given condition \cmd{display \_distance(30:50)}.
Traces which don't have a \cmd{distance}--value at all, are not
put on the list as well as all traces whose distance value is out of
the specified range.  If the lower bound \cmd{<start>}\ is not
specified, all traces with a distance smaller than 50$^\circ$ are
selected (\cmd{display \_distance(:50)}).  The same holds for the
upper bound \cmd{<end>}, \cmd{display \_distance(30:)}\ displays
traces with a distance larger than 30$^\circ$.  The expression
\cmd{\_distance(30)}\ selects traces which have a distance value
of exactly 30$^\circ$.  For real--valued entries this is usually
senseless, but for other entry types like \shi{string}\ or
\shi{char}\ or \shi{time}\ it may be useful.  So the command
\cmd{display \_comp(z)}\ displays only z--components.  But notice
that SH converts the command line to uppercase letters by default.
This is important for string and character comparisons.  You can
negate the selection conditions by putting a "\verb'~'"--character
after the info entry name.  This means, the expression
\verb'_distance~(30:50)'\ selects all traces whose distance value
is {\em not\/} in the range between 30$^\circ$ and 50$^\circ$.
Some instructive examples of commands using list expressions are
given below.
\smallskip

\noindent
\cmd{del 4-6,8,11,15-20}\\
Delete traces number 4, 5, 6, 8, 11, 15, 16, 17, 18, 19, 20.
\smallskip

\noindent
\cmd{hide all}\\
Hide all traces.
\smallskip

\noindent
\cmd{display h:all}\\
Display all hidden traces starting at position 1.
\smallskip

\noindent
\cmd{sum \_azimuth(0:180)}\\
Sum traces whose azimuth value is between 0 and 180$^\circ$.
\smallskip

\noindent
\verb'del _comp~(z)'\\
Delete all traces which are not z--components.
\smallskip

\noindent
\cmd{zoom/rel \_station(bji) 2}\\
Enlarge the amplitude of all traces of station BJI by a factor of 2.
\smallskip

\noindent
\cmd{hide \_magnitude(:5.8)}\\
Hide all traces with a magnitude smaller than 5.8.



\section{Trace Filtering}

Application of filters is a very important step in analysing
seismic traces.  It is performed in almost any case of data
processing.  For this reason the related commands are explained
here in more detail in addition to the interactive help texts.
SH knows three different kind of filters:  FFT--filters,
recursive filters and tabulated filters.  Besides these SH
can perform operations related to filtering, like
Hilbert--transformation, attenuation and computation of minimum delay
signals from given signals or autocorrelations.

\subsection{FFT Filters}

This is the most common way of trace filtering.  A copy of the trace
is transformed into frequency domain using the FFT (Fast Fourier
Transformation) algorithm.  If the number of samples is not a
power of 2, zeroes are appended to the trace automatically.
In the frequency domain the trace is multiplied by the filter
function which is given as poles and zeroes (In detail, the
trace is multiplied with the first zero, then divided by the
first pole, then multiplied with the second zero, then divided
by the second pole and so on until all zeroes and all poles
are used).  After this process the trace is transformed back
into time domain and displayed on screen (if the redraw is
enabled).  SH stores the currently used filter (or a cascade of
filters) in memory.  That means, before applying the filter operation
you have to read in an FFT filter file, containing poles and zeroes
and a normalization factor.  This is a text file which you can
create with any text editor or with utility programs.  An example of
such a filter file is given here:

\begin{verbatim}
! file WWSSN_SP.FLF
!      ============
!
! WWSSN-LP FFT filter (including GRF restitution)
! H(s) = ( P(s,h0,w0) / P(s,h1,w1) ) * ( w2*w2 / P(s,h2,w2) )
! where P(s,h,w) := s*s + 2*h*w*s + w*w
! h0 = 0.707,  t0 = 2*pi/w0 = 20.0
! h1 = 0.67,   t1 = 2*pi/w1 = 1.05
! h2 = 0.55,   t2 = 2*pi/w2 = 0.75
!
1357913578
1
70.18385353
2
(-0.2221106,-0.2221777)
(-0.2221106,0.2221777)
4
(-4.009271,-4.442279)
(-4.009271,4.442279)
(-4.607669,-6.996659)
(-4.607669,6.996659)
\end{verbatim}

\noindent
At the beginning of the file may be (better: should be) comment
lines.  Comment lines start with an exclamation sign "\exm{!}" (no
preceding blanks !).  The first line after the comments is a magic
number and must be \exm{1357913578}.  This identifies the file
to be an SH filter file.  The next line is another ID number,
specifying the filter type.  For FFT filter files this is \exm{1}.
The next line contains a normalization constant which is multiplied
to the filtered trace.  Then follows, again in a separate line, the
number of zeroes of the filter.  In the example file two zeroes are
given.  A pair of complex conjugated zeroes counts as two
zeroes and both of them must be specified in consecutive, separate
lines.  The real and imaginary part are separated by a comma and
enclosed in parantheses.  Even real numbers must be entered in this
format.  The poles are given similarly.  First the number
of poles, then the complex poles in separate lines.  Please note
that blank lines are not permitted at any place in the file.

The given filter is a simulation filter for a WWSSN--SP seismomemter
which needs a velocity--proportional record of an STS--1 instrument
as input.  The filter file \exm{WWSSN\_SP.FLF}\ is read into memory
by the command \cmd{fili f wwssn\_sp}.  The default extension
\exm{.FLF}\ may be omitted.  The filter files are searched in
the current directory and in the filter library (see section
\ref{sec:SHLibs}\ about default paths in SH).  The \exm{f}\ as
second parameter denotes that the file contains an FFT filter.  It
is possible to specify more than one filter file on this command.
For example, the input \cmd{fili f f1 f2 f3}\ reads in the three
FFT files \exm{F1.FLF}, \exm{F2.FLF}\ and \exm{F3.FLF}.  The
filters are applied to the trace one after the other (filter
cascade).  If the qualifier \exm{/compress}\ is specified, all
filters are concatenated to a single filter and zero--valued
poles and zeroes are shortened if possible (non--zero values are
not shortened, I'm sorry for this).  Once the filter (cascade) is
read in, it remains in memory until another \cmd{fili}--command
replaces it.  Each filter process uses the filter(s)
read in by the most recent \cmd{fili}--command.  To apply the
filter to the first three traces on display, type in
\exm{filter f 1-3}.  Again, the \exm{f}--parameter denotes an FFT
filter process.  After execution of this command, three new traces
appear on top of the display containing the filter output.
FFT filters are quite slow if the input traces are very long
(several ten thousands of samples).  It is possible to set a time
window for the filter process.  If you want to filter only the
time window between 50 and 350 seconds (relative to the time
axis), enter \cmd{filter f 1-3 50 350}.  By default, that means
without parameters 3 and 4, the whole trace is filtered (not
only the part inside the display window).


\subsection{Recursive Filters}

Recursive filters have the advantage that they are faster on long
traces than FFT filters.  The reason is, that output samples are
computed as linear combinations of already existing samples of the
input and the output trace.  Thus the computation time grows
proportional to the number of samples $N$, while the FFT filters
grow proportional to $N\log N$.  On the other hand, recursive filters
have some deficiencies as well.  These are mainly:
\begin{itemize}
\item
   Usually it takes more time to determine the recursive filter
   coefficients than the poles and zeroes for a given transfer
   function.  Particularly if the poles and zeroes are given,
   an FFT filter can be written down immediately, while a recursive
   filter needs a considerable amount of brainwork.
\item
   The filter coefficients depend on the sample rate.  Such a
   recursive filter can be applied only to traces of a fixed
   sample rate.
\item
   On very long--period filters, where many samples are involved
   to compute a new output sample, recursive filters tend to
   numerical instabilities.
\item
   Output traces of recursive filters usually have high--amplitude
   numerical noise at their beginning.
\end{itemize}
As SH supports both filter types it is up to the user to decide
which filter is to be preferred in his application.

The formula used in the recursive filter operations is this:
\[
   r_n = a_0 f_n + a_1 f_{n-1} + \ldots + a_k f_{n-k} -
      b_1 r_{n-1} - b_2 r_{n-2} - \ldots - b_l r_{n-l}
\]
where the $a_i, i=0,\ldots,k$ and the $b_j, j=1,\ldots,l$ are the
filter coefficients.  The input trace is given by the $f$'s and
the output trace by the $r$'s.  This means, the current output
sample $r_n$ is determined by the current input sample $f_n$,
$k$ previous input samples and $l$ previous output samples.
The first $k$ output samples access input samples with indices
smaller than zero, which are not known.  These are assumed to
be zero.  This is the reason for the numerical noise at the beginning
of the output trace.

An SH recursive filter file can contain one or more recursive
filters.  The whole filter file is read in by the command
\cmd{fili}.  An example file \exm{WWSSN\_SP.FLR}\ of a
WWSSN-SP filter is given here:

\begin{verbatim}
! file WWSSN_SP.FLR
!      ============
!
! WWSSN-SP recursive filter
!
1357913578
3
0.05
1.0
3
1.011167
-1.999877
0.9889559
3
1.224691
-1.954563
0.8207457
@
3
0.05
1.0
3
4.5180295e-2
9.0360589e-2
4.5180295e-2
3
1.278993
-1.909639
0.8113681
\end{verbatim}

It again needs as input velocity--proportional records of an STS--1
instrument.  The first lines beginning with "\exm{!}" are comment
lines and are ignored by SH (but not by the human reader of the
file !).  The first line after the comments must be the magic number
\exm{1357913578}.  The next line contains the ID number of
recursive filters and is \exm{3}.  This is followed by the
sample distance (in sec) of the coefficients.  SH stores this value
and permits the user to apply this filter only to traces of this
sample rate.  The next line is a normalization number and is
usually 1.  Then the number of $a$--coefficients is specified, which
is \exm{3}\ in the example file.  That means, this number is followed
by 3 coefficients, namely $a_0$, $a_1$ and $a_2$.  The same holds
for the $b$--coefficients.  The example file gives three $b$'s,
$b_0$, $b_1$ and $b_2$.  Please note that in the filter file
a $b_0$ must be specified which does not appear in the above
formula (in fact, this is the coefficient of the $r_n$ on the left
side of the equation).  SH eliminates $b_0$ after reading the
file by dividing all $a_i, i=0,\ldots,k$ and $b_j, j=1,\ldots,l$ by
$b_0$.  After the
coefficient $b_2$, the first filter in the example file ends.
The next line contains the separation character "\verb'@'",
indicating that another recursive filter is appended.  The second
filter starts with the ID number \exm{3}\ for recursive filters
(there is no more magic number).  Then, again, follows the
sample distance, the normalization, the $a$--coefficients and
the $b$--coefficients.  This example file contains a cascade
of two recursive filters.  You can use up to 5 filters in one
file.  The command \cmd{fili r wwssn\_sp}\ reads the whole filter
file into memory.  The default extension \exm{.FLR}\ may be
omitted.  Recursive filters of a previous
\cmd{fili}--command are replaced by this cascade  (FFT filters
are not affected by this command).  The command \cmd{filter r all}\
applies the filter cascade to all of the traces on display.
The result traces are appended to the top of the display.  As
for FFT filters you can specify a time window by optional
parameters (number 3 and 4).  These denote the lower and upper
bound of the window (in sec) relative to the time axis.


\subsection{Tabulated Filters}

Sometimes a transfer function may be given as a tabulated function
instead of poles and zeroes or it is desirable to create an
acausal filter in the frequency domain.  In such cases you can
use the tabulated filters of SH.  This is a text file containing
the tabulated values of amplitude and phase transfer functions.
These tabulated values may be non--equidistant.
The filter process is performed similar to the FFT filters.
A copy of the input trace in transformed to the frequency domain
and each frequency sample is multiplied by an interpolated value
of the tabulated filter.  The amplitude and phase function of
the filter are both linearly interpolated.  The filtered trace
is transformed back to the time domain and displayed on screen.

A simple example file \exm{TRAPEZ.FLT}\ of a tabulated filter
is listed below.

\begin{verbatim}
! file TRAPEZ.FLT
!      ============
!
! simple tabulated filter
!
1357913578
2
6
0.0  0.0  0.0
0.02 0.0  0.0
0.03 1.0  0.0
0.3  1.0  0.0
0.5  0.0  0.0
100.0 0.0  0.0
\end{verbatim}

The first lines with a "\exm{!}"--character in the first column
are comments.  The first line after the comments is the well--known
magic number \exm{1357913578}.  It is followed by the ID number
\exm{2}\ of tabulated filters.  The next line contains the number
of tabulated filter values (here \exm{6}).  Then follows one line
for each point in the frequency domain.  Every line consists of
three numbers.  First the frequency in Hz, then the amplitude
function and at last the phase function.  The example shows a
trapezoidal amplitude function and a zero phase function.  The
transfer function is flat (amplitude 1) between 0.03\,Hz and 0.3\,Hz.
Designing such a filter you should make sure that the tabulated
function covers at least the area in frequency domain which is
used by the input trace, that is from zero to the nyquist
frequency.  If you fail to do so the filter command will abort
with an error message.  The filter file is read in with the
command \cmd{fili t trapez}.  The default extension \cmd{.FLT}\
may be omitted.  The filter operation is done by \cmd{filter t <list>},
where \cmd{<list>}\ denotes any trace list.  As for the other filters
you can specify a time window in optional parameters number 3 and 4.


\subsection{Special Filters}

SH knows some other filter operations which are used sometimes.
The Hilbert transformation is one of them.  It is applied with
the command \cmd{filter h <list>}, where \cmd{<list>}\ may be any
trace list like \cmd{4-6}\ or \cmd{2}\ or \cmd{all}.  As for any
other filter commands, the input traces remain unchanged and the
new output traces are appended to the top of the window.  You can
restrict this operation to a time window if you specify the lower
and upper bound (in sec, relative to the time axis) as parameter
numbers 3 and 4, respectively.
\smallskip

Another option is the attenuation of a trace by an attenuation
operator
\[
   A(\omega) =  e^{ -\frac{1}{2}\omega t^* +
      i\frac{\omega}{\pi}\ln\frac{\omega}{\omega_N} }
\]
where $t^*$ is the attenuation parameter and $\omega_N$ is the
Nyquist frequency.  $t^*$ is passed to the command as parameter
number 5.  If you don't want to set a time window for the operation,
then you have to enter empty parameters 3 and 4.  The command to
attenuate the first three traces on display by a $t^*$ of 0.5\,s is
\cmd{filter a 1-3;;;0.5}.
\smallskip

One option related to filter operations remains to be mentioned
in this section.  It is the computation of a minimum delay signal.
Input is either an arbitrary time
signal or an autocorrelation.  If a time signal is given, the
command is \cmd{filter m <list>;;;<shift>}, for an autocorrelation
it is \cmd{filter c <list>;;;<shift>}.  \exm{<list>}\ is a trace
list like \exm{4-6}\ or \exm{all}\ and \exm{<shift>}\ is a time
shift in s by which the output trace is shifted to the right.
If you specify 0 for the \exm{<shift>}\ parameter then
the signal on the output trace starts exactly at the beginning of
the trace which is not very convenient in most cases.  Of course,
it is possible to enter a time window in parameters 3 and 4 as
in the other filter commands.


\section{Scaling of Amplitudes}

If more than one trace is in the display window of SH one has to
consider, how to relate the individual trace amplitudes.  If the
records of a 3--component seismometer or traces of a station array
are displayed, everyone prefers to have the same amplification
factors for each trace.  This makes sure that the true amplitude
ratios are shown.  In other cases where several records of the
same event from globally distributed stations are displayed or
where filtered and unfiltered waveforms are to be compared, it is
often more convenient to have the scaling in such a way that each
trace is plotted with the same amplitude.  Also it may be desirable
sometimes to amplify a subset of the displayed traces in amplitude.
SH can manage each of these problems, there exist quite a few
commands to manipulate the trace amplitudes.  This section describes
how to use this subset of commands.

It is useful to know a bit about how SH determines the actual
display amplitudes.  Each trace has two temporary info entries
(\shi{mem-only}\ type, see section \ref{sec:InfoEntries}), called
\shi{norm}\ and \shi{zoom}.  The display amplitude results as
the product of the seismogram amplitude (this is given by the
sample values) and these two real numbers.  The \shi{norm}\ entry is
determined by SH before each redraw, depending on the current
active normalization mode.  The \shi{zoom}\ entry is set by the
user.  Thus exist three different ways to manipulate the trace
amplitudes:
\begin{enumerate}
\item
   Changing the normalization mode using the \cmd{norm}\ command.
   This tells SH how to determine the normalization factor
   (\shi{norm}\ entry) for each trace.
\item
   Changing the zoom factor (\shi{zoom}\ entry) via \cmd{zoom}\
   command.
\item
   Changing the sample values of the trace (commands \cmd{trcfct} and
   \cmd{unit}).  This is somewhat problematic if this is done only
   for reasons of the trace display, because it really changes your
   traces.  If you sum these traces at a later time, they may have
   the wrong weights.  If you write the traces to a file, the
   amplitude changes are saved as well.  Therefore you should prefer
   the commands \cmd{norm}\ and \cmd{zoom}.  The entries \shi{norm}\
   and \shi{zoom}\ are temporary and are not saved in an output file.
   Also, these entries don't have any influence on operations with
   the traces, they affect the display only.
\end{enumerate}

\noindent
{\bf point 1:}
The normalization mode is changed by the \cmd{norm}\ command.  This
command accepts only one parameter, which must be one out of a set
of five short strings:
\begin{deflist}{\exm{aw}}
\item[\exm{af}]
   The normalization factor is the same for each trace and is
   determined as half of the reciprocal value of the maximum
   sample (without sign) of all displayed traces on their total
   length (not only inside the displayed window) using the info
   entries \shi{maxval}\ and \shi{minval}.  That means,
   all traces are normalized to the maximum value on all traces
   within their full length.
\item[\exm{aw}]
   Here the normalization factor also is the same for each trace,
   but the maximum value is determined only within the displayed
   window.  The traces are normalized to the maximum value on all
   traces within the displayed window.  Since the determination
   of the maximum value is done before each redraw and since the
   info entries \shi{maxval}\ and \shi{minval}\ cannot be used
   here, this normalization mode may slow down the program
   considerably, if long traces are to be searched.
\item[\exm{sf}]
   The normalization is determined for each trace separately as
   half of the reciprocal value of the maximum sample (without sign)
   on it's full length (not only inside the displayed window)
   using the info entries \shi{maxval}\ and \shi{minval}.
   That means, all traces have the same display amplitude, if the
   \shi{zoom}--factors are identical.
\item[\exm{sw}]
   Normalization is again determined separately, but now inside
   the current display window, resulting in equal display amplitudes
   for traces with the same zoom factor.  Since the determination
   of the maximum value is done before each redraw and since the
   info entries \shi{maxval}\ and \shi{minval}\ cannot be used
   here, this normalization mode may slow down the program
   considerably, if long traces are to be searched.
\item[\exm{c}]
   All normalization factors are set to 1.  This is useful for
   applications, where different plots must have the same
   amplitude scale.  In all other modes the normalization depends
   on the sample amplitudes of the traces on display.  If you make
   plots of two different data sets you cannot compare the amplitudes
   between the plots.  This problem is solved, if you use the
   \exm{c}--mode and if you always set the same zoom factor and
   if you have always the same number of traces on display.  But
   after switching to the \cmd{c}--mode, you have to be aware that
   the display amplitudes may look like zero (if the sample amplitudes
   are very small) or may be immense (for big sample amplitudes).
   In any case you have to find out appropriate zoom factors by
   yourself.
\end{deflist}
The modes \exm{af}\ and \exm{sf}\ sometimes confuse users, if
large amplitudes exist outside the display window.  In this case
the display amplitudes of some or all traces are very small.
This can be checked either by deleting the time window or by
switching to the modes \exm{aw}\ or \exm{sw}.  But keep in mind
that the latter modes can slow down the program if you deal with
long traces.  The default mode (if not changed in the setup file)
is \exm{af}.
\smallskip

{\bf point 2:}
The zoom factor is changed via the \cmd{zoom}\ command.  The syntax
is \cmd{zoom <list> <factor>}.  Since there is a \exm{<list>}\
parameter, it is possible to scale the traces independently.
\exm{<factor>}\ is copied to the \shi{zoom}\ entry of the specified
traces.  It remains there until it is explicitely changed.
Since the value is copied to the entry, a command \cmd{zoom 1 2}
doesn't change anything, if the first trace is already zoomed
by a factor of 2.  But it is possible to enter relative factors
as well.  This is done by \cmd{zoom/rel 1 2}, which magnifies the
display amplitude of the first trace by a factor of 2, no
matter what the zoom factor currently is.  Traces which are created
by SH operations and which are appended to the top of the display,
get a default zoom factor which is usually 1.  This means, if all
traces on the display have a zoom factor of 5 and a new trace is
created, this appears to be relatively small, because of the default
zoom factor of 1.  However, with the command
\cmd{zoom/default <list> <factor>} you can change the default zoom
factor (and the zoom factor of the traces in \exm{<list>}).

{\bf point 3:}
If all else fails (or seems to be too inconvenient) the samples
itself can be multiplied by a factor.  But as it is mentioned
above, this might result in problems in future operations on
such traces.  You have always to keep in mind that these traces
do not have the original sample values any more.  A subset
\exm{<list>}\ of traces can be multiplied by a number \exm{<r>},
using the command \cmd{trcfct <list> mul <r>}.  On the display
amplitudes this will have the same effect as
\cmd{zoom/rel <list> <r>}.  Therefore this command isn't really
necessary for reasons of display.  More convenient is the
\cmd{unit}\ command.  It determines the absolute maximum (without
sign) of a given list of traces and within a given time window.
This maximum is set to 1 and all other samples of the specified
traces are normalized with respect to this absolute maximum
(the sign remains untouched).  A typical application is, if there
are two or more three--component sets of records on the screen,
with large amplitude differences between the sets.  With
\cmd{unit}\ each set can get the maximum sample amplitude of 1.
This way the sets can be compared without loosing information
about the relative amplitudes within a set.  The syntax of the
\cmd{unit}\ command is \cmd{unit <list> <lo> <hi>}.  \exm{<list>}\
specifies the set of traces to be normalized together.
\exm{<lo>}\ and \exm{<hi>}\ contain the lower and upper bound
of the time window in s (relative to the time axis) where to look
for the absolute maximum.  If these parameters are omitted, the
full traces are used.


\section{Command Procedures}
\label{sec:CmdProc}

The ability of SH to process command procedures is one of it's
most important features.  Most of SH's internal commands
operate on a rather low level.  Therefore SH is very flexible
and can be used for many different and quite special purposes.
On the other hand, more complex operations consist of several
basic instructions and require a considerable amount of keyboard
input.  For the convenience of the user it is therefore often
necessary to combine these low--level instructions and create
more elaborate commands.  Thus the user interface is optimized
for the solution of particular data processing problems.

A command procedure is an editable text file.  Each line of
this file holds a single SH command (and/or comments).  The
commands are processed sequentially until the \cmd{return}\
command is found.  This terminates the execution of the current
procedure and returns to the parent command level which is either
the interactive level or another command procedure.  The end of
file also terminates execution but it creates an error message
and returns to the interactive level in any case.  The command
procedure is called by the name of the command file without
extension which is assumed to be \exm{.SHC}.

\subsection{A first example}

The usage of command procedures will be explained here by
developing an example file which will get more complex step
by step.  The purpose of this command file is to perform a
3--dimensional rotation from the recording Z,N,E--coordinate system
to the local ray system of the P--wave, called L,Q,T--system.
This rotation needs two angles, the azimuth and the angle of
incidence.  For the first example let's assume that there exists
a q--file named \exm{q\_exm}\ which contains a three component
record with the components Z, N and E at the file positions 1, 2
and 3, respectively.  The azimuth and angle of incidence are
known.  Let their values be 207.2$^\circ$ and 19.5$^\circ$,
respectively.  The following procedure \cmd{rotex1}\ reads in the
traces, rotates them and writes the result to an output q-file,
named \exm{q\_out}.

\begin{verbatim}
! file rotex1.shc
!
! version 1 of the 3-dim rotation procedure
! K. Stammler, 15-Apr-92

del all             ! delete possibly existing traces ...
dtw                 ! ... and time window

read q_exm 1-3      ! read in Z,N,E components
rot 1-3 207.2 19.5  ! rotate, create three new traces
write q_out 4-6     ! write result traces to output q-file

return              ! return to parent level
\end{verbatim}

After entering the command \cmd{rotex1}\ this procedure is executed
if SH can find the command file.  SH looks for the file
\exm{rotex1.shc}\ in the current directory and in a common
command directory.  Of course, the q--file \exm{q\_exm}\ has to
be in your current directory, otherwise you have to specify the
directory explicitely.

\subsection{Command Parameters}

In practice the above command procedure is not very useful,
because the azimuth and angle of incidence are given as fixed
values in the file.  These angles are usually different for
each event.  Instead of changing the text each time, it is
much more convenient to pass these values as parameters to
the procedure.  In the command procedure the parameter number
\exm{N}\ is accessed by the expression \exm{\#N}.  So the rotation
command \exm{rot 1-3 207.2 19.5}\ of the procedure should be
replaced by \exm{rot 1-3 \#1 \#2}.  Then the procedure works fine as
long as the user passes the azimuth as the first parameter and
the angle of incidence as the second.  But imagine the case that
the user can't remember the order of the parameters or doesn't
know the parameters at all.  In particular for more complex
procedures with five or ten parameters this problem would be even
worse.  Guessing parameters is very tedious and annoying.
Therefore a command is implemented which prompts the user for
the parameters if he wants to.  Besides that this command
assigns default values to parameters which are left empty in
the calling command line.  For this reason the command is called
\cmd{default}.  The first parameter of the \cmd{default}--command
specifies the number of the parameter, the second contains the
default value.  If this is empty, no default value is assigned.
All following parameters are used as a prompt text if the user
is to be prompted for input.  If the command procedure is called
without any parameters (not even an empty parameter) all parameters
of the procedure are prompted using the given prompt text of the
\cmd{default}--command.  If the user specifies at least one parameter
(even if it is empty), no parameter is prompted.  Instead, every
empty or not specified parameter gets its default value from
the \cmd{default}--command.  The updated version of the rotation
procedure now looks like this:

\begin{verbatim}
! file rotex2.shc
!
! version 2 of the 3-dim rotation procedure
! K. Stammler, 15-Apr-92

default 1 ;;    input q-file        ! input file
default 2 ;;    azimuth             ! no default for azimuth
default 3 0.    angle of incidence  ! default 0 for incidence
default 4 q_out output q-file       ! output file

del all             ! delete possibly existing traces ...
dtw                 ! ... and time window

read #1 1-3         ! read in Z,N,E components
rot 1-3 #2 #3       ! rotate, create three new traces
write #4 4-6        ! write result traces to output q-file

return              ! return to parent level
\end{verbatim}

The command line \exm{rotex2 q\_ex 207.2 19.5 q\_out}\ is now equivalent
to the first version of the procedure \exm{rotex1}.  If you enter
\exm{rotex2 q\_ex 207.2}, the angle of incidence defaults to 0 and
the output file defaults to \exm{q\_out}\ without
any prompting.  The plain command \exm{rotex2}\ without
parameters let SH prompt you for all of the parameters in the
specified order, using the prompts given in the \cmd{default}\
command.  At the third and fourth parameter,
additionally to the prompt text, a default value is offered.
You can accept it just by hitting the return key, otherwise you
have to enter another value.  You should use the
\cmd{default}--command on every parameter in order to supply
the user with information texts about the parameters.  If you do
so, the command procedure behaves exactly like an internal
command.

\subsection{Variables in SH}
\label{sec:Variables}

The recent version of the rotation procedure is not really a
big improvement for the user.  He still needs to know the rotation
angles for each event and has to pass it to the command
procedure.  SH can determine these rotation angles with the
command \cmd{mdir}.  But variables are needed to store the
results of \cmd{mdir}\ and to pass them to \cmd{rot}.

The concept of variables in SH is similar to other programming
languages.  There exist global and local variables.  Local
variables are visible only inside the command procedure where
they were defined.  The local variables are deleted when the
procedure is terminated.  Global variables are visible in any
command level, that means in all command procedures and in the
interactive level.  In order to keep a good programming style
there shouldn't be defined too many global variables (as in any other
language).  Usually the four predefined global variables are
sufficient.  Their names are \exm{g1}, \exm{g2}, \exm{g3}\ and
\exm{ret}.  These variables are needed only to store return values of
command procedures.

All variables must be defined.  Any access to an undefined variable
results in an error message.  The definition command is \cmd{sdef}
(symbol definition).  The first parameter is the name of the
variable to be defined.  An optional second parameter specifies
its initial value.  By default the defined variables are local.
For a global definition the qualifier \cmd{/global}\ is required.
SH variables have no type.  That means in a variable you can store
any information, like integers, floating point numbers, strings and
others.  SH does not check any type information, so you have to be
aware of what you are doing.  In fact, any values are stored as
strings and are converted internally if necessary.  To assign
a value to a variable, it has to be passed to a command which
returns an output value.  In this case a \cmd{\&}--character has
to be placed in front of the variable name, to indicate a write
access.  A read access to a variable is made by a preceding
\cmd{"}--character.  The command parser assumes that any word
in the command line beginning with a double quote is a defined
variable and replaces the expression by its current value.

Now follows version 3 of the rotation procedure
which let the user select a time window to determine the rotation
angles via \cmd{mdir}--command.

\begin{verbatim}
! file rotex3.shc
!
! version 3 of the 3-dim rotation procedure
! K. Stammler, 16-Apr-92

default 1 ;;    input q-file        ! input file
default 2 q_out output q-file       ! output file

sdef azim                 ! define azimuth variable, no init
sdef inci                 ! define angle of incidence, no init

del all                   ! delete possibly existing traces ...
dtw                       ! ... and time window

read #1 1-3               ! read in Z,N,E components
echo select time window   ! message to the user
mdir 1-3 *;;&azim &inci   ! determine angles from time window
rot 1-3 "azim "inci       ! rotate, create three new traces
write #2 4-6              ! write result traces to output q-file

return                    ! return to parent level
\end{verbatim}

In earlier versions of SH \cmd{mdir}\ won't work properly inside
a command procedures if the time window is user--selected
(\verb+*+--parameter).  In this case two additional variables
holding begin and end of the time window must be defined.  Their
values are assigned using the command \cmd{time} twice (example:
\verb+time;;&start+).  The window bounds must be passed to the
command \cmd{mdir}\ by these variables.

\subsection{The Command {\tt calc}}

An important command for assigning values to variables is \cmd{calc}.
It performs simple numerical computations as well as text and time
manipulations.  The general syntax is:
\begin{verbatim}
   calc <type> <outvar> = <operand1> [<op> <operand2>] [<p>]
\end{verbatim}
This fixed structure allows only one operation per command line.
Here the same syntax rules hold as for any other command.  That
means in particular that the "\cmd{=}" character and the operator
\cmd{<op>}\ are treated as ordinary parameters which must be
separated by blanks (or semicolons) from the preceding and following
expressions.  \verb+<type>+ specifies the type of the operation.
This is necessary, because all variables can have any type and
SH needs to know whether there is to compute an integer addition
or a floating point addition or a string concatenation, for example.
\verb+<outfile>+ gives the name of the variable (preceding the
\exm{\&} character) where to store the result of the operation.
If only one operand is specified the instruction performs a plain
assignment of the value of \verb+<operand1>+ to \verb+<outvar>+.
\verb+<p>+ is an additional parameter which is used only in a
few special operations.
\bigskip

\noindent
Valid operators for the type "\exm{i}" (integer) are:
\begin{deflist}{\exm{cnv\_julian}}
\item[\exm{+}]
   Adds the two operands.
\item[\exm{-}]
   Subtracts \verb+<operand2>+ from \verb+<operand1>+.
\item[\exm{*}]
   Multiplies the two operands.
\item[\exm{div}]
   Divides \verb+<operand1>+ by \verb+<operand2>+ without remainder.
\item[\exm{mod}]
   Remainder of the division \verb+<operand1>+ by \verb+<operand2>+.
\end{deflist}
\bigskip

\noindent
Valid operators for the type "\exm{r}" (real) are:
\begin{deflist}{\exm{cnv\_julian}}
\item[\exm{+}]
   Adds the two operands.
\item[\exm{-}]
   Subtracts \verb+<operand2>+ from \verb+<operand1>+.
\item[\exm{*}]
   Multiplies the two operands.
\item[\exm{div}]
   Divides \verb+<operand1>+ by \verb+<operand2>+.
\item[\exm{abs}]
   Removes the sign of \verb'<operand1>'.  No second operand.
\item[\exm{arctan2}]
   Computes the arc tangens of \verb'<operand1>/<operand2>' in
   degrees.  Works correct even if \verb'<operand2>' is equal or
   close to zero.
\item[\exm{power}]
   Takes \verb'<operand1>' to the \verb'<operand2>'--th power.
\end{deflist}
Additionally there exist a number of single argument functions.
The argument is specified by \verb'<operand1>', \verb'<operand2>'
must be empty.  All trigonometric functions (and their inverse)
work with degrees as input (output).  A complete list of the
functions is \exm{sin, cos, tan, arcsin, arccos, arctan,
sinh, cosh, tanh, exp, log, ln, sqrt}.
\bigskip

\noindent
Valid operators for the type "\exm{s}" (string) are:
\begin{deflist}{\exm{cnv\_julian}}
\item[\exm{+}]
   Concatenation of \verb'<operand1>' and \verb'<operand2>'
   with a blank in between.
\item[\exm{parse}]
   Extracts the \verb'<operand2>'--th word from \verb'<operand1>'.
   Separation characters are blanks and semicolons.
   \verb'<operand2>' must be integer.
\item[\exm{extract}]
   Extracts \verb'<p>' characters from the string \verb'<operand1>'
   starting at position \verb'<operand2>'.  The first character
   of \verb'<operand1>' has number 1 (not 0).  \verb'<operand2>' and
   \verb'<p>' must be integer.
\end{deflist}
\bigskip

\noindent
Valid operators for the type "\exm{t}" (time) are:
\begin{deflist}{\exm{cnv\_julian}}
\item[\exm{tdiff}]
   Computes the difference of two absolute time values
   \verb'<operand1>' and \verb'<operand2>'.  The difference is
   given in seconds and stored as a floating point number.
\item[\exm{tadd}]
   Adds \verb'<operand2>' seconds to the absolute time value
   \verb'<operand1>'.  \verb'<operand2>' must be a floating point
   number, negative numbers are accepted.  The result is again an
   absolute time value.
\item[\exm{cnv\_julian}]
   Converts the day number \verb'<operand2>' of the year
   \verb'<operand1>' into day and month.  The two output integers
   are stored in \verb'<outvar>' separated by a blank.
   Both operands must be integer (in an appropriate range).
\item[\exm{make\_time}]
   Converts an absolute time string in numeric format (examples:
   \cmd{"30,\-12,\-85,\-5,\-30,\-20,\-5"}\ or
   \cmd{"85/12/30/5/30/20/5"})
   to the standard format of absolute time
   (\cmd{"30-DEC-1985\_5:30:20.005"}).
\end{deflist}
\bigskip

\noindent
Examples:
\begin{verbatim}
   calc i &cnt = "cnt + 1       ! increments counter cnt
   calc r &num = "a * "b        ! multiplies two floating points
   calc r &num = "x sqrt        ! takes square root of x
   calc s &str = "str parse 1   ! keeps first word only in str
   calc s &str = "x extract 5 3 ! extracts chars 5 to 7 from x
   calc t &dif = ^p-onset(3) tdiff ^start(3)
                                ! computes time offset of P onset
                                ! relative to start of trace 3
\end{verbatim}

With this command the rotation procedure can run completely
without user interaction.  Supposition is, however, that the
P--wave onset is inserted to the q--file header.  There exists
a predefined info entry, called \exm{p-onset}.  It should contain
the absolute time of the P onset.  The user still has to select
the P--wave times for all his events like in the previous versions
of the procedure, but now he has to do it only once.  Moreover, this
process is decoupled from the rotation procedure and the gathered
information can be used for other applications as well.  The
insertion of the onset time is done by two commands if the event
is already on screen.  The command \cmd{time \&g1}\ let the user
select the onset time by graphic cursor and stores the result
in the (global) variable \cmd{g1}\ which is predefined.  Of course,
you can use any other variable as well.  The second command is
\cmd{set/file all p-onset "g1}, which inserts the determined time
to the headers of all traces on display.  For details see command
\cmd{set}.  Another info entry used in the command procedure is the
start time of the trace.
This is also an absolute time, the name of this (predefined) entry
is \cmd{start}.  Now follows the automated version 4 of the rotation
procedure.  It writes the L--, Q-- and T--components to separate
files.

\begin{verbatim}
! file rotex4.shc
!
! version 4 of the 3-dim rotation procedure
! K. Stammler, 20-Apr-92

default 1 ;;    input q-file          ! input file
default 2 q_out output q-file prefix  ! output file prefix
default 3 2.    window width (sec)    ! time window for
                                      ! determining angles

sdef azim                 ! azimuth
sdef inci                 ! angle of incidence
sdef start                ! start of time window (relative time)
sdef end                  ! end of time window (relative time)

del all                   ! delete possibly existing traces ...
dtw                       ! ... and time window

read #1 1-3               ! read in Z,N,E components
calc t &start = ^p-onset(1) tdiff ^start(1)
                          ! determine relative time of P-onset
calc r &end = "start + #3 ! get end of time window
mdir 1-3 "start "end &azim &inci
                          ! determine angles in computed window
rot 1-3 "azim "inci       ! rotate, create three new traces
write |#2|_l| 4           ! write L-component to L-file
write |#2|_q| 5           ! write Q-component to Q-file
write |#2|_t| 6           ! write T-component to T-file

return                    ! return to parent level
\end{verbatim}


\subsection{Loops}

Version 4 of the rotation procedure is already well developed.
It enables the user to process several 3--component records.  But
suppose he has 100 events stored on q--files.  He would have to
type in 100 similar command lines.  Even if he would write
a command procedure calling \cmd{rotex4}\ 100 times, he would have
to insert the 100 file names to the procedure text.  It is more
convenient to store the 100 file names in a separate file (this
may be created with a directory command redirected to a file) and
process all files in a command loop.

For loop structures SH provides two commands, \cmd{goto}\ and
\cmd{if}.  \cmd{goto}\ has the syntax
\begin{verbatim}
   goto <label>
\end{verbatim}
This command requests SH to jump to the specified label
\exm{<label>}.  \exm{<label>}\ is a text string with a colon
":" at the end.  SH looks for this label inside the current
command procedure.  If the qualifier \exm{/forward}\ is specified
it starts looking at the current position, otherwise it rewinds
the command file before searching.  A label is any text string at
the beginning of a command line that ends with a colon.  Labels are
ignored in ordinary execution of command procedures.  They only serve
as jump addresses.  Label lines must not contain any executable
commands.

The command \cmd{if}\ performs the conditional execution of an
instruction.  It's syntax is
\begin{verbatim}
   if  <e1> <cmp> <e2>  <cmd> [<label>]
\end{verbatim}
The sequence \cmd{<e1> <cmd> <e2>}\ is a compare operation between
two expressions \cmd{<e1>}\ and \cmd{<e2>}\ with a compare
operator \cmd{<cmp>}.  If the result of this comparison is true,
the instruction \cmd{<cmd>}\ is executed.  Only two SH instructions
are permitted in place of \cmd{<cmd>}, namely \cmd{return}\ and
\cmd{goto}.  \cmd{return}\ terminates the current command
procedure and \cmd{goto}\ jumps to a specified label \cmd{<label>}.

The compare operator \cmd{<cmp>}\ specifies the actual comparison
and the type of the expressions \cmd{<e1>}\ and \cmd{<e2>}.  Valid
operators are:
\begin{description}
\item[\exm{eqi}]  \ integer comparison, \cmd{<e1>} is equal to \cmd{<e2>}
\item[\exm{nei}]  \ integer comparison, \cmd{<e1>} is not equal to \cmd{<e2>}
\item[\exm{lei}]  \ integer comparison, \cmd{<e1>} is less or equal \cmd{<e2>}
\item[\exm{lti}]  \ integer comparison, \cmd{<e1>} is less than \cmd{<e2>}
\item[\exm{gei}]  \ integer comparison, \cmd{<e1>} is greater or equal \cmd{<e2>}
\item[\exm{gti}]  \ integer comparison, \cmd{<e1>} is greater than \cmd{<e2>}
\item[\exm{eqr}]  \ float comparison, \cmd{<e1>} is equal to \cmd{<e2>}
\item[\exm{ner}]  \ float comparison, \cmd{<e1>} is not equal to \cmd{<e2>}
\item[\exm{ler}]  \ float comparison, \cmd{<e1>} is less or equal \cmd{<e2>}
\item[\exm{ltr}]  \ float comparison, \cmd{<e1>} is less than \cmd{<e2>}
\item[\exm{ger}]  \ float comparison, \cmd{<e1>} is greater or equal \cmd{<e2>}
\item[\exm{gtr}]  \ float comparison, \cmd{<e1>} is greater than \cmd{<e2>}
\item[\exm{eqs}]  \ string comparison, \cmd{<e1>} is equal to \cmd{<e2>}
\item[\exm{nes}]  \ string comparison, \cmd{<e1>} is not equal to \cmd{<e2>}
\end{description}

With this supplement it is possible to write a command procedure
to apply \cmd{rotex4}\ to all q--files listed in a list file.

\begin{verbatim}
! file rotex5.shc
!
! version 5 of the 3-dim rotation procedure
! K. Stammler, 21-Apr-92

default 1 ;;      q-file list ! name of list file
default 2 1       first file  ! first file to process
default 3 %#1(0)  last file   ! last file, default is last line
default 4 q_out   output file ! output file prefix
default 5 2.      time window ! width of window

sdef cnt #2                   ! file counter init. to start line
sdef qfile                    ! name of current q-file

nr                            ! switch off redraw, incr. speed
loop_start:                   ! start of loop (only label)
   if  "cnt gti #3  goto/forward loop_exit:
                              ! exit if counter exceeds last file
   calc s &qfile = %#1("cnt)  ! get current q-file from list
   echo processing file "qfile! message to the user
   rotex4 "qfile #4 #5        ! call rotex4
   calc i &cnt = "cnt + 1     ! increment counter
goto loop_start:              ! repeat loop
loop_exit:                    ! loop exit
rd                            ! switch on redraw again

return                        ! return to parent level
\end{verbatim}

This example file also demonstrates nesting of command procedures,
because this procedure, \cmd{rotex5}, calls another procedure,
\cmd{rotex4}.  Nesting is permitted up to a level of nine calls,
which is very likely sufficient in all applications.  In the
above example the \cmd{nr}\ command switches off the automatic
redraw, which increases execution speed.  But don't forget to
switch it on again (\cmd{rd}).



\subsection{Execution Flags}
\label{sec:ExecFlags}

Execution flags control some details in the processing of a
command procedure (some affect the interactive level as well).
Examples are the suppression of error
messages, abortion of procedures on errors and tracing through
a procedure.  Such features can be switched on and off by the
\cmd{switch}\ command.  It's syntax is \cmd{switch <flag> <on/off>}.
\exm{<flag>}\ is the name of the execution flag and \exm{<on/off>}\
is either \exm{on}\ or \exm{off}.  A list of all flags is given
below:

\begin{deflist}{\exm{cmderrstop}}
\item[\exm{cmderrstop}]  X--flag.
   Controls whether the command procedure is aborted on errors.
   If it is switched on, SH returns to the interactive level
   after displaying the error message.  Usually this option is
   switched on.  After such an error the command procedure should
   be corrected.  Be careful in switching off this flag, because
   you may get lot of messages of subsequent errors.  In the worst
   case you may create an infinite loop in the command procedure
   and you will have to abort SH in a very crude way.
   Default is \exm{on}.
\item[\exm{sherrstop}]  A--flag (Abort).
   Controls whether SH is terminated after an error occurred.  This
   is useful only if SH is executed in batch mode. Default is
   \exm{off}.
\item[\exm{verify}]  V--flag.
   Controls whether a verification of each command is printed before
   it is executed.  With this option the translation of the command
   lines can be checked, since the command is printed after
   the translation process.  Default is \exm{off}.
\item[\exm{echo}]  E--flag.
   Controls whether all commands are echoed (without translation)
   before execution.  Default is \exm{off}.
\item[\exm{step}]  T--flag (Trace).
   Controls whether SH prompts the user for keyboard input
   (\exm{<Return>}--key) before each command.  Useful in
   combination with V--flag.  Default is \exm{off}.
\item[\exm{protocol}]  P--flag.
   Controls whether the interactive commands of the user are logged
   in the protocol file of SH.  Default is \exm{on}.
\item[\exm{capcnv}]  C--flag.
   Controls whether each input character is converted to uppercase.
   Default is \exm{on}.
\item[\exm{noerrmsg}]  Q--flag (Quiet).
   If this switch is on, no error messages (and no warning bell)
   is printed.  Useful in combination with X--flag if possibly
   occurring errors are handled by the command procedure.
   Default is \exm{off}.
\item[\exm{chatty}]  I--flag (Info).
   Some commands give an explaining text if this flag is switched
   on (like command \cmd{sum}).  This info text may be switched off
   if the internal structure of the command procedure should be
   hidden from the user.  Default is \exm{on}.
\end{deflist}

By default, the switches are changed only locally within the
current command procedure.  The flags are reset when the procedure
terminates.  To change flags globally, the qualifier \exm{/global}\
must be entered on the \cmd{switch}\ command.  Then the switches
remain in the specified state until they are changed again by a
\cmd{switch/global}\ command.  Besides the \exm{switch}\ command
there exists another method
to change the flags.  When a command procedure is called, valued
qualifiers \exm{/flags}, \exm{/flags+}\ or \exm{/flags-}\ may
be applied.  The value of the qualifier is a string consisting
of one or more flag characters.  The flag characters are given
in the above list.  \exm{/flags}\ sets all specified flags to
\exm{on}, all others to \exm{off}.  \exm{/flags+}\ sets all
specified flags to \exm{on}, all others remain unchanged.
\exm{/flags-}\ set all specified flags to \exm{off}, all others
remain unchanged.  The flags are changed only locally within the
called command procedure, unless the \exm{/global}\ qualifier
is also specified.  With the \exm{/global}\ qualifier the
changes affect all child levels of the called procedure.



\subsection{Debugging Tools}

Most of the execution flags mentioned in the last section are not
very important for most users.  But two of them are very useful
if a command procedure contains a mistake which must be
detected.  With the V--flag all translated commands can be listed
in the dialog window.  Usually the last command before the
error message caused the error.  In some more complicated cases
the whole procedure must be traced step by step.  Additionally
to the V--flag, the T--flag lets the user check each command
carefully by prompting for keyboard input (\exm{<Return>}--key)
before each command.  If you want to debug the command procedure
\exm{bugproc}, you have to enter \cmd{bugproc/flags+=v}.
Possibly existing command parameters can be passed as usual:
\cmd{bugproc/flags+=v p1 p2}.  This call enables the command
verification.  If you want to switch on trace mode as well, the
command is \cmd{bugproc/flags+=vt}.  The flags are set only in
\exm{bugproc}, if \exm{bugproc}\ calls another command procedure,
this is processed without verifying and trace mode.  To make SH
tracing all child levels as well, type in
\cmd{bugproc/flags+=vt/global}.  If the error is detected,
it may be tedious to continue tracing until the procedure
terminates.  An input of "\verb'@'" terminates tracing (and
the command procedure).  Of course, the V-- and T--flags may
be set by the \cmd{switch}\ command as well, but then you have
to change the text of the command procedure.

In general when a command procedured is aborted by an error a
{\em status report file\/} is created in the SH scratch directory.
The file name is a concatenation of the SH Session ID and the
string \exm{ERR.STX}.  The Session ID always starts with
\exm{SH}\ or \exm{SH\$}\ or \exm{SH\_}, followed by a random number.
The random number is necessary to make each SH session ID unique on a
multi--tasking operating system.  All scratch files of SH start with
the Session ID string to avoid interference of different SH
sessions with each other.  The same holds for the status report files.
A typical name for a status file on VMS is \exm{SH\$2354\_ERR.STX}.
It contains error number, error message and the calling chain
(all levels) of the command procedures including the line
numbers where the error occurred.  Also given are the execution
flags, all defined local (named {\em symbols set 0\/}) and global
(named {\em symbols set 1\/}) variables and all parameters and
qualifiers passed to the procedure.  This information is very
helpful for debugging command procedures.




\section{Internal Variables}
\label{sec:InternVar}

Internal variables are a bit different from the local and global
variables that the user can define with \cmd{sdef}.  They are
a fixed set and are defined internally by SH.  Also they do not
contain information about traces as the trace info entries do.
Internal variables are read--only for the user.  Some of them
reflect SH status parameters, like \exm{\$dsptrcs}\ or
\exm{\$tottrcs}, others contain fixed values, like all the character
variables (\exm{\$blank}, \exm{\$dollar}, \ldots).  All names
start with a "\exm{\$}"--character and are replaced by the command
interpreter by it's current value.  A complete list is:

\begin{deflist}{\exm{\$timeaxisstyle}}
\item[\exm{\$dsptrcs}]
   Number of traces in the display window.
\item[\exm{\$tottrcs}]
   Total number of traces in memory.
\item[\exm{\$status}]
   Return status of the last executed command.  Zero means
   successful completion, any other value is an error number.
\item[\exm{\$systime}]
   Current system date and time string.  This is not yet implemented
   in all operating systems.
\item[\exm{\$version}]
   Returns current version of SH.
\item[\exm{\$dsp\_x}]
   Returns the lower bound of the current time window (set by command
   \cmd{stw}).
\item[\exm{\$dsp\_xmax}]
   Returns the upper bound of the current time window (set by command
   \cmd{stw}).
\item[\exm{\$dsp\_w}]
   Returns the width of the current time window (set by command
   \cmd{stw}).
\item[\exm{\$dsp\_y}]
   Returns the lower bound of the current vertical window
   (y--window, set by command \cmd{syw}).
\item[\exm{\$dsp\_ymax}]
   Returns the upper bound of the current vertical window
   (y--window, set by command \cmd{syw}).
\item[\exm{\$dsp\_h}]
   Returns the width of the current vertical window (y--window,
   set by command \cmd{syw}).
\item[\exm{\$titlestyle}]
   Returns style block number of the title text lines.  For
   additional information about attribute blocks, see section
   \ref{sec:AttribBlocks}.
\item[\exm{\$trcinfostyle}]
   Returns style block number of the trace info text
\item[\exm{\$zerotrcstyle}]
   Returns style block number of the line attributes of zero traces
\item[\exm{\$timeaxisstyle}]
   Returns style block number of the line style of the time axis
   and the label text
\item[\exm{\$markstyle}]
   Returns style block number of the marker lines
\item[\exm{\$x}]
   This is a special variable, pointing to the currently drawn
   trace on the display.  It is useful only in connection with
   the command \cmd{trctxt}.
\end{deflist}

\noindent
Additionally there exist character values which return special
characters.  These variables may be indexed, like info entries.
The index number in parantheses is interpreted as a repeat counter.
For example, the expression \exm{\$blank(10)}\ returns a string
of ten blanks.  You need the \exm{\$blank}\ variable if you want
to pass a string parameter which contains blanks.  If you would
specify the blanks in the parameter directly, SH wouldn't regard
the string as a single parameter and the parameters would be passed
incorrectly.  The correct parameter specification of an example
parameter "\exm{text with blanks}" is:
\begin{verbatim}
   |text|$blank|with|$blank|blanks|
\end{verbatim}
Besides \exm{\$blank}\ other special characters are available.
The names are self--explaining, here follows only a list of their
names: \exm{\$exclamation, \$quotes, \$dollar, \$percent,
\$hat, \$bar, \$slash, \$number}.  You can create any ASCII
character with a \exm{\$hexchar??}--expression.  The two
question marks stand for the two digits of the character's
ASCII code (hexadecimal).  For example, two consecutive horizontal
tabulators are created by \exm{\$hexchar09(2)}.


\section{Output Attributes}
\label{sec:AttribBlocks}

Every output, text or lines, to the graphics window of SH is
controlled by attribute blocks.  An attribute block stores
information, how thick a line is drawn, which line style and
which colour is used.  Also it determines the text attributes
like character height, font and colour.  There are several
attribute blocks available, the exact number depends on the
actual graphics interface (at least 10).  Most of these blocks
are for free use, but some are reserved for special output items.

To change an entry in an attribute block you have to use the
command \cmd{fct setstyle <block> <item> <value>}.  \cmd{<block>}\
specifies the attribute block number, \cmd{<item>}\ the item to
be changed and \cmd{<value>}\ it's new value.  Valid items are:
\begin{deflist}{\exm{linestyle}}
\item[\exm{linewidth}]
   Set line width.  \exm{<value>}\ specifies the line width in
   pixels.
\item[\exm{linestyle}]
   Set line style.  \exm{<value>}\ is an integer number specifying
   the line style.  Default style is 0 (continuous line).  All
   other styles are dashed or dotted lines.
\item[\exm{color}]
   Set line colour.  \exm{<value>}\ is an expression defining the
   colour.  In X--Window and VWS there must be specified three
   real numbers between 0 and 1, separated by commans (no blanks).
   These numbers are the red, green and blue fractions of the
   output colour.  Usually, there are colour files \exm{RED.STX},
   \exm{YELLOW.STX}\, \exm{GREEN.STX}\ and so on in the
   \exm{globals}--directory of SH.
\item[\exm{charsize}]
   Set the size of characters.  \exm{<value>}\ specifies the
   character height in units of window height.  For example, the
   labelling of the time axis has usually the size 0.02.
\item[\exm{font}]
   Set character font.  \exm{<value>}\ contains the font number.
\item[\exm{wrmode}]
   Set writing mode.  \exm{<value>}\ is either \exm{replace}\ or
   \exm{xor}.  \exm{replace}\ is the default mode.
\end{deflist}

As mentioned above, some attribute blocks are reserved for special
use.  The actual numbers of these blocks may be different in each
implementation and they may change in future versions of SH.
Therefore are internal variables defined, which contain the attribute
block numbers of various output items.  A complete list of these
internal variables can be found in section \ref{sec:InternVar}.  An
example is \exm{\$markstyle}, which is the style block number of
the vertical trace markers (lines) and their labelling (text).
Usually the marker lines are red (on coloured screens).  If you
want to change the line width to five pixels, you have to enter
\cmd{fct setstyle \$markstyle linewidth 5}.  All following
markers are then thick red lines.  Of course, you can change
the colour to blue: \cmd{fct setstyle \$markstyle color \%blue}.
In this command the colour expression is read from the file
\exm{BLUE.STX}\ in the \exm{globals}\ directory of SH.  Such colour
files exist for the most common colours.  It is possible to
add other colour files if necessary.

The traces on display do not have fixed attribute blocks.  By
default, they use block number 0 for output.  All traces have
an info entry, called \exm{attrib}, which contains the current
block number.  So each trace can use a different attribute
block if it is necessary and if enough blocks are available.
The entry value is changed by the \cmd{set}\ command, as any
other info entry.  To change the colour of the traces 4-6 to red,
you need two commands.  First set an attribute block (for example
number 1) to red colour: \cmd{fct setstyle 1 color \%red}.  Then
assign this block to the selected traces: \cmd{set 4-6 attrib 1}.
After the next redraw (command \exm{rd}), these traces will be
displayed in red colour.  At the end of this section some
examples are given.
\smallskip

\noindent
\cmd{fct setstyle \$timeaxisstyle linewidth 3}\\
Change line width of time axis to thicker lines.
\smallskip

\noindent
\cmd{fct setstyle \$markstyle wrmode xor}\\
Change trace markers to XOR--mode.  Now the markers can be removed
if they are drawn a second time at the same position.
\smallskip

\noindent
\cmd{fct setstyle 3 color \%yellow}\\
Set colour of attribute block number 3 to yellow.
\smallskip

\noindent
\cmd{set all attrib 3}\\
Use attribute block 3 for all traces on display.
\smallskip

\noindent
\cmd{set 1,4,6 attrib 3}\\
Use attribute block 3 for traces 1, 4 and 6.
\smallskip

\noindent
\cmd{set \_magnitude(7.0:) attrib 3}\\
Use attribute block 3 for all traces with a magnitude larger than 7.0.



\section{SH Paths and Input Files}
\label{sec:SHLibs}

Many operations like filter input or calling a command procedure
require input files.  These input files are searched in the
current directory and, if they are not found there, in special
default directories.  SH knows several separate directories for
help files, for command procedures, filters and others.  The actual
paths are usually defined in the startup file of SH.  It is very
unlikely that they have to be changed after SH is installed
properly, nevertheless it is useful to explain the path commands
here.  The command syntax is \cmd{fct path <name> <dir>}.
\cmd{<name>}\ specifies the name of the directory to be set and
\cmd{<dir>}\ contains the actual path.  Examples for \cmd{<path>}\
are \exm{/home/sh/filter/}\ in UNIX and \cmd{disk1:[main.sh.command]}\
in VMS.  It follows a list of valid directory names and a short
description of their content.

\begin{deflist}{\exm{command}}
\item[\exm{scratch}]
   Directory for scratch output files like hardcopy files, protocol
   files, error files and so on.  This directory must not be changed
   within a command procedure, because otherwise the status flags
   and local parameters of the parent command level cannot be restored
   properly.  This is due to the fact, that when a command procedure
   is called, the status, local variables and parameters of the parent
   level are saved to a temporary file in the scratch directory
   (\exm{.SSV}--files).  If the scratch directory is changed within
   this procedure this file cannot be found any more after the
   procedure has terminated.
\item[\exm{help}]
   Contains all help files about built--in commands and command
   procedures.  Help files have the extension \exm{.HLP}.  New files
   can be added to this directory if new command procedures are to
   be documented.
\item[\exm{command}]
   Directory of common command procedures.  Procedures in this
   directory are available to all users of SH.
\item[\exm{globals}]
   Default directory for access to input text files, that means
   if the file \exm{text.stx}\ in an expression \exm{\%text("cnt)}\
   cannot be found in the current directory, it is searched in this
   directory.
\item[\exm{filter}]
   Filter files of any type (FFT, recursive and tabulated) in this
   directory are available to every user of SH.
\item[\exm{errors}]
   Contains the error messages of SH.  The corresponding text of an
   error number 1820, for example, is found in the file
   \exm{ERR\_1800.MSG}\ in line 20.
\item[\exm{inputs}]
   Directory for input data files like travel time tables
   (\exm{.TTT}--files) of various phases.
\end{deflist}



\section{Travel Times}

There is a utility routine implemented in SH to read travel times
of various phases from travel time tables.  These travel times may
be used to mark the theoretical arrival times in the seismograms.
The travel time tables are assumed to be in the \exm{input}\
directory of SH.  This default directory may be changed by the
command \cmd{fct tt\_table <path>}, where \cmd{<path>}\ contains
the actual directory path of the travel time tables.  Of course,
the tables have to be in a special format.

Each seismic phase is stored in a separate file, called
\exm{.TTT}--file.  The name of the file is given by the name
of the phase (for example \exm{P.TTT}\ for P--phases or
\exm{PKP.TTT}\ for PKP--phases).  Since many operating systems are
not case--sensitive, phases like pP or PcP need a special
convention.  All lowercase letters in the phase names are converted
to uppercase with a preceding "V"--character.  That means,
the travel times of the phase PcP are read from a file
\exm{PVCP.TTT}, pP from \exm{VPP.TTT}\ and so on.  In extreme
cases like Pdiff (\exm{PVDVIVFVF.TTT}) this is a bit clumsy, but
this doesn't happen too often.  The \exm{.TTT}--files itself
are ASCII files of the following format:

\begin{verbatim}
! travel time table for PP-waves
! 
! created by Ray Buland's program TTIMES
TTT
distance bounds
27.0 180.0
depth steps
15 0.0 50.0 100.0 150.0 200.0 250.0 300.0 350.0 ...
 27.0 000000 000000 000000 391.70 388.42 385.42 382.77 380.48 ...
 28.0 000000 409.94 406.33 402.81 399.53 396.53 393.87 391.57 ...
 29.0 426.71 421.05 417.44 413.92 410.64 407.63 404.96 402.65 ...
 30.0 437.82 432.16 428.55 425.02 421.73 418.72 416.04 413.71 ...
 31.0 448.93 443.27 439.65 436.12 432.82 429.79 427.10 424.75 ...
    :
    :
\end{verbatim}

The first lines beginning with "\exm{!}" are ignored by SH.  The
first line after the comments contains an identification string
\exm{TTT}\ (uppercase letters).  The next line after this is
ignored and is used for comments.  The following line specifies
the distance bounds in which valid travel time information is
available in the file.  Then follows again a comment line.  The
next information is the number $N$ of depth steps (\exm{15}\ in the
example file).  Then the depth values (in km) of all $N$ depths must
be specified.  After this header information the travel time data
are listed.  Each line contains travel times of one epicentral
distance.  The distance value in degrees is the first number
in the line.  It is followed by $N$ travel times in seconds, one
for each depth step defined above.  Travel time of illegal
combinations of depth and distance are given as zeroes.

If the travel time for given distance and depth is requested
SH opens the file and reads the header information.  If distance
or depth are out of range the routine is aborted and an error message
is displayed.  Otherwise SH scans all lines until the distance found
is greater than the requested distance (it doesn't check EOF,
so the distance bounds of the header must be definitely correct).
It takes this and the previous line and interpolates the travel
times linearly in distance and depth.  The resulting time is returned.
To call the travel time utility, enter the command
\cmd{call travel <phase> <distance> <depth> <result>}.  \cmd{<phase>}\
specifies the seismic phase.  If the name contains lowercase letters,
you have to use the "V"--convention or you turn off the automatic
case conversion (then you have to enter uppercase commands and
keywords).  \cmd{<distance>}\ and \cmd{<depth>}\ are the distance
and depth of the event in degrees and km, respectively.  \cmd{<result>}
is the name of the output variable (with a preceding
"\exm{\&}"--character) where to store the resulting travel time.

\noindent
Examples:
\smallskip

\noindent
\cmd{call travel p 98.0 33.0 \&tt}\\
After this call the variable \exm{tt}\ contains the resulting
travel time of P (\exm{tt}\ must be a defined variable).  It may be
used in subsequent commands as input.  To inspect the variable
use the \cmd{echo}\ command as for any other expression:
\cmd{echo "tt}.
\smallskip

\noindent
\cmd{call travel vpp 50.0 0.0 \&tt}\\
Computation of pP travel time.  An equivalent command sequence
is given below.
\smallskip

\noindent
\cmd{switch capcnv off}\\
\cmd{CALL TRAVEL pP 50.0 0.0 \&TT}\\
\cmd{SWITCH CAPCNV ON}\\
Returns also pP travel time in variable \exm{tt}.
\smallskip

A short command procedure \exm{markphase.shc}\ explains how
theoretical arrival times of an arbitrary phase are marked
on a seismogram.  Supposition is that the trace is read from
a q--file with given informations about origin time (entry
\exm{origin}), distance (entry \exm{distance}) and depth
(entry \exm{depth}).  If this is not the case, additional parameters
to the command procedure must be defined, supplying it with the
requested information.

\begin{verbatim}
! file MARKPHASE.SHC
!
! marks phase on a selected trace
! K. Stammler, 27-Apr-92
!
default 1 1    trace number        ! which trace to mark
switch capcnv off
DEFAULT 2 P    phase name          ! case sensitive prompt
SWITCH CAPCNV ON

sdef tt                            ! output travel time

call travel #2 ^distance(#1) ^depth(#1) &tt
                                   ! compute travel time
calc t &tt = ^origin(#1) tadd "tt  ! get absolute arrival time
mark/abs/label=#2 #1 "tt           ! mark time position

return
\end{verbatim}

Important is that the second parameter is only case--sensitive
if the procedure is called without parameters and all parameters
are prompted.  A call like \cmd{markphase 1 PcP}\ won't work,
because first the command parser converts all letters to uppercase
and then passes the parameters to the procedure \cmd{markphase}.
In this case you have to use the "V"--convention: \cmd{markphase
1 pvcp}.



\section{Event Locations and Beams}

If data from a station array are available, SH is able to determine
azimuth and slowness of a selected phase.  Conversely, it computes
beam traces if azimuth and slowness are given.  These operations
need to know at which stations the individual seismograms are
recorded and where the stations are located.  The first information
is easy to get, because there exists a predefined info entry
\exm{station}\ specifying the recording station.  The second part,
the location of the stations is independend from individual
seismograms and is therefore not available in info entries.  SH
uses a special text file which lists station information for
many different stations.  This file is accessed if one of the
above operations (commands \cmd{locate}\ and \cmd{beam}) is
performed.  If it is not found, the operation is aborted.  To tell
SH where to find the location file, use the command
\cmd{fct locfile <file>}.  \cmd{<file>}\ specifies the complete
filename including path and extension.  An example file
\exm{statloc.dat}\ shows the format of such a file.

\begin{verbatim}
GRA1 +49.6918877 +11.2217202 1 -21.923 +37.862   Graefenb...
GRB1 +49.3913475 +11.6519526 1  +9.085  +4.308   Graefenb...   
GRC1 +48.9961681 +11.5213504 1  -0.775 -39.662   Graefenb...
GRA2 +49.6552079 +11.3594439 1 -11.985 +33.668   Graefenb...
GRA3 +49.7622038 +11.3186951 1 -14.815 +45.618   Graefenb...
GRA4 +49.5654029 +11.4358711 1  -6.575 +23.668   Graefenb...
GRB2 +49.2709252 +11.6699661 1 +10.145  -9.162   Graefenb...
GRB3 +49.3435419 +11.8059826 1 +20.165  -1.112   Graefenb...
GRB4 +49.4689373 +11.5608463 1  +2.445 +12.858   Graefenb...
GRB5 +49.1121310 +11.6767332 1 +10.785 -26.972   Graefenb...
GRC2 +48.8675675 +11.3755426 1 -11.595 -53.902   Graefenb...
GRC3 +48.8901739 +11.5858216 1  +3.805 -51.472   Graefenb...
GRC4 +49.0867465 +11.5262720 1  -0.355 -29.602   Graefenb...
WET  +49.14      +12.88      0   0.      0.      FRG, Wetzell
BFO  +48.3        +8.3       0   0.      0.      FRG, Black F...
HAM  +53.46       +9.92      0   0.      0.      FRG, Hamburg
CLZ  +51.8       +10.4       0   0.      0.      FRG, Clausthal
BRL  +52.2       +13.5       0   0.      0.      FRG, Berlin
FUR  +48.16      +11.28      0   0.      0.      FRG, Fuerst...
TNS  +50.22       +8.45      0   0.      0.      FRG, Taunus
\end{verbatim}

Each line contains one station, starting with the name of the
station (usually in uppercase letters), the latitude and the
longitude in degrees.  The next number is an array ID code (integer),
followed by two floating point numbers specifying the relative
array positions of the stations in km.  All following text in the
line may be used as a comment.  Please make sure that no line exceeds
the length of 132 characters.  By the array ID code the stations
can be grouped to arrays.  Each array must have a unique array
number (is \exm{1}\ for the GRF--array in the example file).
Stations which do not belong to an array must get the array code 0.
Each array station must also have relative coordinates (these are
the last two numbers in the line).  Stations with an array code
of 0 may have zero values for the relative coordinates.
The command \cmd{locate}\ uses the relative positioning rather
than latitude and longitude if all traces passed are recorded
at the same station array (which means all have the same non--zero
array code).  This can be prevented by specifying the
\exm{/noarray}--qualifier on the \cmd{locate}\ command.

To compute azimuth and slowness of a phase, \cmd{locate}\ needs it's
travel time differences at each station.  If you just enter
\cmd{locate}\ (or \cmd{locate/noarray}) you have to pick the phase
at each station by graphic cursor (exit the selection by
pressing the "\exm{E}"--key).  \cmd{locate}\ uses the plain
time differences between the picks, it does not account for
time--shifted traces.  So please make sure that all traces are
correctly positioned in time with reference to each other.  If all
traces have the same start time, this can be achieved by resetting
all time shifts to zero by the command \cmd{set all t-origin 0}.
If the traces have different start times you need the additional
command \cmd{shift all time\_al}\ to align the traces in time.
The results of the computation are displayed on screen or are
copied to output variables if specified.  The general syntax is
\cmd{locate <list> <azim> <inci> <azim-err> <inci-err>}.  The
first parameter \cmd{<list>}\ is not used in this computation
mode and should be empty.  All other parameters specify output
variables to store results.

The \cmd{beam}\ command acts inversely and applies time shifts
to the traces, determined from the station locations and from
given slowness and azimuth.  The calling syntax is
\cmd{beam <list> <azimuth> <slowness>}.  \cmd{<list>}\ specifies
which traces are shifted (typically \cmd{all}) and \cmd{<azimuth>}\
and \cmd{<slowness>}\ are the (back--)azimuth in degrees and
slowness in s/deg.  \cmd{beam}\ applies relative shifts, like the
\cmd{shift}\ command.  If you apply \cmd{beam}\ twice you get
twice the time shift at each trace.  To align traces in time
before the \cmd{beam}\ command, apply the commands given above.
To obtain a beam trace you have to sum the \cmd{beam}ed traces.
A typical command sequence is:
\begin{verbatim}
set all t-origin 0
shift all time_al
beam all 330.0 5.4
sum all
\end{verbatim}
It creates the beam trace on top of the display.  If this beam
process is applied for many different slownesses at a fixed
azimuth (or reversely for a set of azimuths at a fixed slowness)
you get many summation traces which are comparable to a vespagram.
This is what the library routine \cmd{vespa}\ does.  The parameters
are \cmd{vespa <slo-start> <slo-end> <slo-step> <azimuth> <power>}.
\cmd{<slo-start>}\ and \cmd{<slo-end>}\ specify the slowness
interval, the step size is given by \cmd{<slo-step>}.  For each
slowness step one beam trace is computed.  The fixed azimuth is given
in \cmd{<azimuth>}.  \cmd{<power>}\ determines the order of the
$N$-th Root Process applied in the summation.  The \cmd{vespa}\
command uses all traces on display for beaming.  After execution
the input traces are hidden, on display are the resulting beam
traces.  The info entry \cmd{comment}\ contains the actual
slowness value for each trace.  It is convenient to change the
trace info text to the \exm{comment}--entry (command
\verb'trctxt ^comment($x)').



\section{Notes for UNIX Versions of SH}

The SH program was originally developed on ATARI ST/TT computers
running TOS as operating system and GEM as graphics interface.
Since most parts of SH are written machine--independend it was quite
easy to export the program to VAX/VMS (after implementation of
a VWS and an X--Window interface).  The implementation on UNIX,
however, shows some peculiarities.  This is mainly due to two
features of UNIX, namely the occurrence of slashes "/" in filenames
and case--sensivity.  The slashes are used in SH (as in VMS) to
indicate command qualifiers (like the hyphen in UNIX).  Therefore
in calls to the operating system cannot be used any slashes.
This is very annoying if filenames have to be passed.  This
problem can be solved by the use of the internal variable
\exm{\$slash}\ (see section \ref{sec:InternVar}) in combination
with concatenation expressions (\verb'|$slash|home|$slash|sh|$slash|'\
instead of \exm{/home/sh/}).  This is very clumsy sometimes as you
can see in the example.  Another possibility is to change the
qualifier character to a backslash "$\backslash$".  Then you can
use slashes without problems, but all existing command procedures
in the library must be changed in this way.  I admit that both
solutions are not perfect.
Similar problems arise concerning the case--sensitivity of UNIX.
By default, SH converts every input line to uppercase letters
before translating it.  This automatic case conversion can be
switched off (see section \ref{sec:ExecFlags}), but then you have
to enter all command verbs and keywords in uppercase letters.
This is also not a convenient solution of this problem.  I'm
still thinking about these things and I hope I can optimize the
UNIX interface in the near future.

Please note, that these peculiarities affect only calls to the
operating system (command \cmd{system}).  All other commands
behave exactly like in other implementations.



% planned sections
%
%\section{Special Commands}
%
%\section{The Startup File}
%
%\section{Installing SH}
%
%\section{Station Locations}

\end{document}
